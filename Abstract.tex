\chapter{Zusammenfassung}
Der schwach elektrische Fisch \emph{Apteronotus albifrons} (Wei�stirnmesserfisch) nimmt seine Umwelt mittels aktiver Elektroortung wahr. Durch die Entladung seines elektrischen Organs, die als EOD (\emph{Electric Organ Discharge}) bezeichnet wird, generiert er ein schwaches elektrisches Feld. Dieses �ndert sich, wenn es beispielsweise auf ein Objekt trifft, dessen Widerstand sich vom umgebenden Wasser unterscheidet, oder wenn es zu einer Interferenz mit dem elektrischen Feld eines anderen Individuums kommt. Diese �nderung nimmt \emph{Apteronotus albifrons} mit Elektrorezeptoren in seiner Haut wahr. Lange gingen Forscher davon aus, dass die Elektrorezeptoren dabei scharf auf die EOD-Frequenz des Individuums getunt seien. Die Tiere w�rden demnach nur Frequenzen nahe ihrer eigenen registrieren. Neue, noch unver�ffentlichte, elektrophysiologische Studien weisen jedoch darauf hin, dass \emph{Apteronotus albifrons} auch Frequenzen weit au�erhalb des scharf gefassten Tunings detektieren kann. Das zeigte sich darin, dass im Fischhirn P-Units und die Pyramidenzellen des ELL auf Frequenzen, die mehrere hundert Hertz von der fischeigenen entfernt waren, antworteten. Diese Bachelorarbeit besch�ftigt sich daher mit der Frage, ob \emph{Apteronotus albifrons} aktiv auf die in den P-Units und in den Pyramidenzellen des ELL vorhandenen Informationen zugreifen kann. Daf�r wurde ein Verhaltensexperiment durchgef�hrt, in dem die Versuchstiere mittels Konditionierung darauf trainiert wurden, zwischen einem belohnten (S+) und einem unbelohnten (S-) Stimulus zu unterscheiden. Die beiden Reize unterschieden sich lediglich darin, dass einer von ihnen eine Frequenz enthielt, die au�erhalb des scharf gefassten Tunings lag. Trifft die Hypothese zu, dass \emph{Apteronotus albifrons} auch Frequenzen wahrnehmen kann, die mehrere hundert Hertz von seiner eigenen entfernt sind, sollten die Versuchstiere die beiden Stimuli also unterscheiden k�nnen. Andernfalls m�ssten die Reize von den Fischen als identisch und somit ununterscheidbar wahrgenommen werden. Der Ausgang des Verhaltensexperiments war nicht eindeutig. Nur eines der Versuchstiere zeigte einen entsprechenden Trend und w�hlte den belohnten Stimulus signifikant h�ufiger aus. Um eine gesicherte Aussage treffen zu k�nnen, sind deshalb weitere Experimente notwendig.
