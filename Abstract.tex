\chapter{Zusammenfassung}
Der schwach elektrische Fisch \emph{Apteronotus albifrons} (Wei�stirnmesserfisch) nimmt seine Umwelt mittels aktiver Elektroortung wahr. Hierf�r generiert er mit seinem elektrischen Organ ein schwach elektrisches Feld. Die Entladung des elektrischen Organs wird EOD ('Electric Organ Discharge') genannt. �ndert sich das schwach elektrische Feld des Fisches entweder durch Eindringen eines Objekts mit einem anderen Widerstand als das umgebende Wasser oder durch Interferenz mit dem elektrischen Feld eines anderen Individuums, so nimmt \emph{Apteronotus albifrons} dies mittels Elektrorezeptoren auf seiner Hautoberfl�che wahr. Die Elektrorezeptoren weisen dabei ein Tuning auf eine bestimmte Frequenzbandbreite auf. Bisherige Forschungsarbeiten gingen davon aus, dass das Tuning der Elektrorezeptoren scharf auf die EOD Frequenz des Individuums abgestimmt sei. Die Tiere w�rden demnach Frequenzen im Bereich ihrer eigenen EOD Frequenz wahrnehmen und Frequenzen weit au�erhalb der eigenen EOD Frequenzen nicht wahrnehmen k�nnen. Jede Spezies der schwach elektrischen Fische h�tte somit einen artspezifischen Kommunikationskanal. Neuere elektrophysiologische Studien weisen nun allerdings darauf hin, dass \emph{Apteronotus albifrons} physiologisch dazu in der Lage ist, auch Frequenzen weit au�erhalb des scharf gefassten Tunings zu detektieren. So antworteten P-Units und die Pyramidenzellen des ELL auf weit von der fischeigenen EOD Frequenz entfernte Frequenzen. Dies Arbeit besch�ftigte sich deshalb mit der Frage, ob \emph{Apteronotus albifrons} die Information �ber die Frequenzen, welche in den P-Units und in den Pyramidenzellen des ELL der Tiere vorhanden ist, auch aktiv nutzen kann.
Hierf�r wurde ein Verhaltensexperiment durchgef�hrt, bei welchem die Versuchstiere mittels Konditionierung zwischen einem belohnten (S+) und einem unbelohnten (S-) Stimulus ausw�hlen sollten. Bei einem der Stimuli handelte es sich dabei um ein rein sinusf�rmiges Signal, welches in seiner Frequenz im Bereich des scharf gefassten Tunings lag. Bei dem anderen Stimulus handelte es sich um ein Signal, welches aus zwei kombinierten Sinssschwingungen bestand. Wobei die Frequenz der ersten Sinussschwingung 
der Frequenz des rein sinusf�rmigen Signals entsprach. Die Frequenz der zweiten Sinusschwingung hingegen lag weit ausserhalb des scharf gefassten Tunings. Sollten die Fische nicht in der Lage dazu sein, die Information �ber diese Frequenz weit au�erhalb des Tunings zu verarbeiten und aktiv zu nutzen, w�re zu erwarten, dass der S+ und der S- Stimulus nicht unterschieden werden kann. 
Der Ausgang des Verhaltensexperiment war nicht eindeutig. Es ergab sich zwar ein Trend dazu, dass es den Tieren m�glich ist, die Signale zu unterscheiden, was auf eine aktive Nutzung der in den Pyramidenzellen des ELL und in den P-Units vorhandenen Information hinweist, dennoch sind weitere Experimente f�r eine gesicherte Aussage notwendig.