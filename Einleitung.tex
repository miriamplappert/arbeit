\chapter{Einleitung}

\section{Was sind schwach elektrische Fische?}

gymnotiformes und mormyriformes

aktive elektroortung ? ? passive Elektroortung

kommunikation



Apteronotus albifrons

wei�stirnmesserfisch

Bild von der Art

neugierig/ aktiv



\section{Elektrisches Organ und EOD}

Elektrocyten im schwanzbereich (vgl batterie)


bei apteronotus albifrons nerv�s


Eod = electric organ discharch

bild von eod kurve?



\section{Rezeptoren (ampull�r und tuber�s)}

ampull�r vgl haie ? passive elektroortung
tuber�s ? einzigartig?, aktive elektroortung
p- und t-units
hopkins: tuning von elektrorezeptoren, jeder fisch eigene frequenz
sinusschwingungen basiston und erste harmonische erkl�ren



BILD

(Quelle: http://www.scielo.br/img/revistas/ni/v9n3/05f04.jpg)

fourrier ? Eigenmannia Schwingung l�sst sich in einzelne Sinusschwingungen Unterteilen ? Powerspektrum: Grundschwingung der Frequnez = 1.Harmonische, 
schwingung der doppelten Frequenz = 1. Oberton



	+ 	

S+ (einzelne Sinusschwingung)    S- (eigenmanniaartiges Signal ( Grundschwingung) + 1. Oberton)

\section{Hypothese/ Ziel der Arbeit}
Apteronotus albifrons ist dazu in der Lage ein Signal welches als Basiston eine geringe Frequenz zur eigenen EOD Frequenz hat und ein Signal welches aus einem deutlich tieferen basisfrequenz und einem sinuston als erste harmonische aufgebaut ist zu unterscheiden



soll durch konditionierung getestet werden. Paper Konditionierung?
